% suppress the page number and headers/footers.
\pagestyle{empty}
\pagenumbering{roman}

\begin{titlepage}
  \begin{center}
    \vspace*{1.0cm}

    \Huge
    {\bf Dynamical systems in spiking neuromorphic hardware}

    \vspace*{1.0cm}

    \normalsize
    by \\

    \vspace*{1.0cm}

    \Large
    Aaron Russell Voelker \\

    \vspace*{3.0cm}

    \normalsize
    A thesis \\
    presented to the University of Waterloo \\
    in fulfillment of the \\
    thesis requirement for the degree of \\
    Doctor of Philosophy \\
    in \\
    Computer Science \\

    \vspace*{2.0cm}

    Waterloo, Ontario, Canada, 2019 \\

    \vspace*{1.0cm}

    \copyright\ Aaron R. Voelker 2019 \\
  \end{center}
\end{titlepage}

% no headers, but yes page numbers (starting from ii.)
\pagestyle{plain}
\setcounter{page}{2}

\cleardoublepage
\phantomsection
\addcontentsline{toc}{chapter}{Author's Declaration}

\noindent
This thesis consists of material all of which I authored or co-authored: see Statement of Contributions included in the thesis. This is a true copy of the thesis, including any required final revisions, as accepted by my examiners.

\bigskip

\noindent
I understand that my thesis may be made electronically available to
the public.

\cleardoublepage

\phantomsection
\addcontentsline{toc}{chapter}{Statement of Contributions}
\begin{center}\textbf{Statement of Contributions}\end{center}

\noindent
Yang Voelker provided the artwork for Figure~\ref{fig:architectures} as well as the brain illustration at the end of the front matter (the winning T-shirt design for the 2018 Telluride Neuromorphic Workshop).
Figure~\ref{fig:touch-network} was designed in collaboration with Ken E. Friedl.
Derivations from Lemma~\ref{lemma:coord-transform}, section~\ref{sec:nonlinear-extensions}, and~\ref{sec:mismatch}, extend work in collaboration with my co-authors, Kwabena Boahen and Terrence C. Stewart, both of whom provided important steps in accounting for pulse-extended double-exponential synapses with time-constant mismatch.

\cleardoublepage
