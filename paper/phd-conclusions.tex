\chapter{Conclusions}
\label{chapt:conclusions}

We have discussed a number of theoretical and practical results involving the synthesis of dynamical systems, spiking neural networks, and neuromorphic hardware.
We now summarize our main contributions in order.

First, section~\ref{sec:sub-principles} observes a number of computational ``sub-principles'' that follow from the adoption of the NEF's three principles.
Arbitrary network topologies reduce to a single recurrently connected layer with sparse encoders, decoders with block structure isomorphic to the original graph structure, and a partitioning of time-constants.
Heterogeneous dynamical primitives form the basis for neural computation, and may be expressed within a unified language to facilitate mathematical analyses that leverage the interchangeability and composibility of such primitives.
Chaotic strange attractors emerge from even the simplest spiking implementations of such dynamical systems.
The NEF represents state-vectors by linearly projecting them onto the postsynaptic currents of neurons,  independent of any considerations of what it means to use a ``rate code'' or a ``spike-time code''.
Likewise, these state-vectors represent frequency content that grows linearly with the relative amount of energy required to drive the synapses of each postsynaptic neuron, for a fixed level of precision.

Second, section~\ref{sec:nef-suitability} addresses the suitability of NEF as a framework for compiling SNNs onto neuromorphic hardware.
Correctness is guaranteed by Theorem~\ref{thm:correctness}, which provides a novel proof of the scaling in NEF precision, conditioned upon a specific criteria that characterizes the distribution of neural states.
Scalability is guaranteed by the previous theorem, in conjunction with a number of prior observations made about time, space, and energy requirements that carefully consider the physical dimensions of relevant quantities (Table~\ref{tab:scalability}).
Completeness is provisioned by the Turing-completeness of dynamical systems, which justifies our assertion that spiking neural networks---trained with the NEF to obey some dynamics---represent powerful models of computation.
Robustness is ensured by a volume of prior work, together with our observation that the NEF is robust to white noise bounded by the diameter of its neural basin of attraction.
Extensibility is demonstrated by a large number of Nengo backends supporting a variety of seemingly disparate architectures, in addition to the extensions summarized in point four below.

Third, section~\ref{sec:dynamics-language} provides a number of novel perspectives on various algorithms realized by dynamical systems, in contexts that do not traditionally take a dynamical systems-based approach at the state-space level.
In particular, winner-take-all networks may be given their inputs sequentially, in which case its solution reduces to that of a dynamical system -- as does the case when all inputs are provided simultaneously.
Unsupervised learning of encoders is a dynamical system, local to each synapse, that can be used to learn fixed-points memorizing its encoded vectors. 
Supervised learning of decoders is likewise a dynamical system, that can be unified with system-level dynamics, and subsequently exploited to implement higher-order transfer functions with only a single spiking neuron (see Lemma~\ref{lemma:pes-dynamics} and Theorem~\ref{thm:pes-filtered}).
Lastly, we observe that a large variety of important routines in linear algebra and gradient-based optimization problems may be cast as dynamical systems that resolve the correct solution over time.

Fourth, section~\ref{sec:synaptic-extensions} exposes several theoretical extensions to the third principle of the Neural Engineering Framework, enabling spiking networks to leverage the computations of higher-order synapses.
We thoroughly characterize linear dynamical systems using the transfer function representation, which allows for axonal spike-delays, such as those in Loihi, to be harnessed, while appealing to Lemma~\ref{lemma:coord-transform} to prove the most general case in Theorem~\ref{thm:general-linear}.
Nonlinear extensions are supported by Theorems~\ref{thm:p3cont-nonlinear} and~\ref{thm:p3disc-nonlinear}, and culminate in an application that exploits hetereogeneous mixed-signal synapses in Braindrop.
We then demonstrate that heterogeneous dynamical primitives may in principle be exploited to accurately classify surface textures using a tactile robotic fingertip.
And finally we extend the NEF's optimization problem to simultaneously solve for decoding weights and the optimal linear filter in the context of Principle~3.

Fifth, section~\ref{sec:neurons} considers extensions of the NEF to various neuron model.
In particular, the role of adaptive neurons and biophysical neurons are characterized as decomposing a transfer function into increasingly-sophisticated model representations, and we provide a conceptual roadmap for how progress can be made within the NEF.
Poisson-spiking models are compared to regular-spiking and adaptive neuron models in the context of Principle~1 providing results consistent with Theorem~\ref{thm:correctness} -- namely, that uniform neural states guarantee the scaling of precision, even at arbitrarily high representational frequencies.
This validates our theory, and recommends that Nengo should consider the role of neural-state distributions very carefully in its abstractions.



We have discussed two main theoretical results.
The first provides a method for accurately implementing continuous-time delays in recurrent spiking neural networks.
This begins with a model description of the delay system, and ends with a finite-dimensional representation of the input's history that is mapped onto the dynamics of the synapse.
The second provides a method for harnessing a broad class of synapse models in spiking neural networks, while improving the accuracy of such networks compared to standard NEF implementations.
These extensions are validated in the context of the delay network.

Our extensions to the NEF significantly enhance the framework in two ways.
First, it allows those deploying the NEF on neuromorphics to improve the accuracy of their systems given the higher-order dynamics of mixed-analog-digital synapses~\citep{voelker2017iscas, voelker2017neuromorphic}.
Second, it advances our understanding of the effects of additional biological constraints, including finite rise-times and pure time-delays due to action potential propagation.
Not only can these more sophisticated synapse models be accounted for, but they may be harnessed to directly improve the network-level performance of certain systems.

We exploited this extension to show that it can improve the accuracy of discrete-time simulations of continuous neural dynamics.
We also demonstrated that it can provide accurate implementations of delay networks with a variety of synapse models, allowing systematic exploration of the relationship between synapse- and network-level dynamics.
Finally we suggested that these methods provide new insights into the observed temporal properties of individual cell activity.
Specifically we showed that time cell responses during a delay task are well-approximated by a delay network constructed using these methods.
This same delay network nonlinearly encodes the history of an input stimulus across the delay interval (i.e.,~ a rolling window) by compressing it into a $q$-dimensional state, with length scaling as $\bigoh{ \frac{q}{f} }$, where $f$ is the input frequency.

While we have focused our attention on delay networks in particular, our framework applies to any linear time-invariant system.
As well, though we have not shown it here, as with the original NEF formulation these methods also apply to nonlinear systems.
As a result, these methods characterize a very broad class of combinations of synapse- and network-level spiking dynamical neural networks.

\section{Future Directions}

Many important questions still remain concerning the interactions between Principles~1,~2, and~3.
While the error in our transformations scale as $\bigoh{ \frac{1}{\sqrt{n}} }$ due to independent spiking, it has been shown that near-instantaneous feedback may be used to collaboratively distribute these spikes and scale the error as $\bigoh{ \frac{1}{n} }$~\citep{boerlin2013predictive, thalmeier2016learning}.
This reduction in error has potentially dramatic consequences for the efficiency and scalability of neuromorphics by reducing total spike traffic~\citep{boahen2017neuromorph}.
However, it is currently unclear whether this approach can be applied to a more biologically plausible setting (e.g.,~using neurons with refractory periods) while retaining this linear scaling property.
Similarly, we wish to characterize the network-level effects of spike-rate adaptation, especially at higher input frequencies, in order to understand the computations that are most accurately supported by more detailed neuron models.
This will likely involve extending our work to account for nonlinear dynamical primitives and subsequently harness their effects (e.g.,~bifurcations) to improve certain classes of computations.

Delay network to store and replay episode memories (temporal semantic pointer)

Venn diagram showing intersection between biology, hardware, and what is useful?

Make useful: dendritic computation, adaptation.

Encode higher-frequency information.

Encode general temporal features (e.g., extending delay network to other basis functions).

More principled energy-minimizing network construction methods

\subsection{Energy Minimization}

``Differential encoding'' (adaptation)?

Spike-thinning (accumulator on BrainDrop)

Improving scaling by spreading out the spikes (inspired by Den\`eve)

Interneurons on Loihi

Other ideas: attentional routing (inhibition to the source). Dynamically adjusting firing rates from feedback error. Better automated selection of tuning curves (e.g., finding thresholds latent within the decomposition of functions)

\TODO{spike delays for minimizing PSC variability, like in Deneve's methods}
