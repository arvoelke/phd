\phantomsection
\addcontentsline{toc}{chapter}{Abstract}
\begin{center}\textbf{Abstract}\end{center}

Dynamical systems are powerful machines representing universal models of computation.
Algorithms that perceive stimuli, remember, learn from feedback, plan sequences of actions, and coordinate complex behavioural responses -- may all be formulated using coupled sets of nonlinear differential equations.
The Neural Engineering Framework~(NEF) and software ecosystem, Nengo, provide a general recipe for expressing such model descriptions, and compiling them onto recurrently connected spiking neural networks -- akin to a programming language for spiking models of computation.
%The distributed activity across such networks realize a corresponding set of algorithms
%Nengo has been used to train the world's largest functioning simulation of the human brain, and to map the same network onto one of the world's largest asynchronous spike-based computing architectures, SpiNNaker.
We analyze the theory driving the success of this framework, and expose several core principles underpinning its correctness, scalability, completeness, robustness, and extensibility.
In particular, we derive a number of theoretical extensions to the framework that enable it to systematically leverage the dynamical computations of physical devices on both digital and analog neuromorphic hardware architectures.
At the same time, we expose a novel set of spiking algorithms that recruit an optimal nonlinear encoding of time, which we call the Delay Network~(DN).
Backpropagation across stacked layers of DNs dramatically outperforms stacked Long Short-Term Memory~(LSTM) networks---a state-of-the-art deep recurrent architecture---in terms of accuracy and training time, on a continuous-time memory task, and a chaotic time-series prediction benchmark.
The basic component of this network is shown to function on state-of-the-art spiking neuromorphic hardware: Braindrop and Loihi, approaching the energy-efficiency of the human brain in the former case, and the robustness of conventional computation in the latter case.

\cleardoublepage
