\chapter{Delay Networks}
\label{chapt:delays}

This section is taken from \citet{voelker2018}.

A fundamental dynamical system that has not yet been discussed within the NEF literature is the continuous-time delay line of $\theta$ seconds,\footnote{
\citet{voelker2015computing} proposed a NEF model, but did not include detailed analysis or extensions.
% Here, we elaborate on this network in considerable detail.
}
expressed as:
\begin{align} \label{eq:time-delay}
y(t) = (u \ast \delta_{-\theta})(t) = u(t - \theta)\text{,} \quad \theta > 0 \text{,}
\end{align}
where $\delta_{-\theta}$ denotes a Dirac delta function shifted backwards in time by $\theta$.
This system takes a time-varying scalar signal, $u(t)$, and outputs a purely delayed version, $u(t - \theta)$.
The task of computing this function both accurately and efficiently in a biologically plausible, spiking, dynamical network, is a significant theoretical challenge, that, to our knowledge, has previously remained unsolved.

The continuous-time delay is worthy of detailed consideration for several reasons.
First, it is non-trivial to implement using continuous-time spiking dynamical primitives.
Specifically, equation~\ref{eq:time-delay} requires that we maintain a \emph{rolling window} of length $\theta$ (i.e.,~the history of $u(t)$, going $\theta$ seconds back in time).
Thus computing a delay of $\theta$ seconds is just as hard as computing every delay of length $\theta'$, for all $0 \le \theta' \le \theta$.
Since any finite interval of $\mathbb{R}$ contains an uncountably-infinite number of points, an exact solution for arbitrary $u(t)$ requires that we maintain an uncountably-infinite amount of information in memory.
Second, the delay provides us with a window of input history from which to compute arbitrary nonlinear functions across time.
For instance, the spectrogram of a signal may be computed by a nonlinear combination of delays, as may any finite impulse response~(FIR) filter.
Third, delays introduce a rich set of interesting dynamics into large-scale neural models, including: oscillatory bumps, traveling waves, lurching waves, standing waves, aperiodic regimes, and regimes of multistability~\citep{roxin2005role}.
Fourth, a delay line can be coupled with a single nonlinearity to construct a network displaying many of the same benefits as Reservoir Computing~\citep{appeltant2011information}.
Finally, examining the specific case of the continuous-time delay introduces several methods and concepts we employ in generally extending the NEF (section~\ref{sec:extensions}).

\section{Conventional Approaches}

Reservoir computing / FORCE / LSTMs?

\section{Optimal NEF Solution}
\label{sec:nef-delay}

As it is impossible in practice (i.e.,~given finite-order continuous-time resources) to implement an arbitrary delay, we will instead approximate $u(t)$ as a low-frequency signal, or, equivalently, approximate equation~\ref{eq:time-delay} as a low-dimensional system expanded about the zeroth frequency in the \emph{Laplace domain}.
We begin by transforming equation~\ref{eq:time-delay} into the Laplace domain, $\mathcal{L} \left\{ y(t) \right\} = \mathcal{L} \left\{ u(t) \right\} \mathcal{L} \left\{ \delta_{-\theta}(t) \right\}$, and then using the fact that $\mathcal{L} \left\{ \delta_{-\theta}(t) \right\} = e^{-\theta s}$ to obtain:
\begin{align} \label{eq:tf-delay}
F(s) := \frac{\mathcal{L} \left\{ y(t) \right\}}{\mathcal{L} \left\{ u(t) \right\}} = e^{-\theta s} \text{,}
\end{align}
where $F(s)$ is known as the \emph{transfer function} of the system, defined as the ratio of the Laplace transform of the output to the Laplace transform of the input.
Equation~\ref{eq:tf-delay} should be understood as an equivalent way of expressing equation~\ref{eq:time-delay} in the Laplace domain, where the variable $s$ denotes a complex frequency. 
Thus far, we have only described the transfer function that we would like the network to implement. %while $t$ is non-negative in the time domain.
% Unimportant reminder? The transfer function is the Laplace transform of system's impulse response.

The state-space model discussed in section~\ref{sec:principle3} (equation~\ref{eq:lti}) is related to its transfer function by the following:
\begin{align} \label{eq:ss2tf}
F(s) = C(sI - A)^{-1}B + D \text{.}
\end{align}
Conversely, a transfer function can be converted into a state-space model satisfying equation~\ref{eq:ss2tf} \emph{if and only if} it can be written as a proper ratio of finite polynomials in $s$~\citep{brogan1982modern}.
The ratio is proper when the degree of the numerator does not exceed that of the denominator.
In such a case, the output will not depend on future input, and so the system is \emph{causal}.
The degree of the denominator corresponds to the dimensionality of the state-vector, and therefore must be finite.
These two conditions align with physically realistic constraints where time may only progress forward, and neural resources are limited.

However, the pure delay (equation~\ref{eq:tf-delay}) has infinite order when expressed as a ratio of polynomials in $s$, and so the system is irrational, or infinite-dimensional.
Consequently, no finite state-space model will exist for $F(s)$, which formalizes our previous intuition that an exact solution is impossible for finite, continuous-time systems.
To overcome this, we must approximate the irrational transfer function $e^{-\theta s}$ as a proper ratio of finite-order polynomials.
We do so using its \emph{Pad\'e approximants}---the coefficients of a Taylor series extended to the ratio of two polynomials---expanded about $s=0$~\citep{Pade1892, vajta2000some}:
\begin{equation} \label{eq:pade}
\begin{aligned}
\left[ p/q \right] e^{-\theta s} &= \frac{\mathcal{B}_{p}(-\theta s)}{\mathcal{B}_{q}(\theta s)} \text{,} \\
\quad \mathcal{B}_m(s) &:= \sum_{i=0}^m \begin{pmatrix}m \\ i\end{pmatrix} \frac{(p + q - i)!}{(p + q)!} s^i \text{.}
\end{aligned}
\end{equation}
This provides the transfer function of order $p$ in the numerator and order $q$ in the denominator that optimally approximates equation~\ref{eq:tf-delay} for low-frequency inputs (i.e.,~up to order $p + q$).

After choosing $0 \le p \le q$, we may numerically find a state-space model $(A\text{,}\, B\text{,}\, C\text{,}\, D)$ that satisfies equation~\ref{eq:ss2tf} using standard methods,\footnote{
For instance, the function \texttt{tf2ss} in MATLAB or SciPy.}
and then map this system onto the synapse using Principle~3.
%This works independently of the chosen simulation time-step using equation~\ref{eq:p3discrete}.
However, na\"ively applying this conversion leads to numerical issues in the representation (i.e.,~dimensions that grow exponentially in magnitude), due in part to the factorials in equation~\ref{eq:pade}.

To overcome this problem, we derive an equivalent yet normalized state-space model that we have not encountered elsewhere.
We do so for the case of $p = q - 1$, since this provides the best approximation to the step-response. We symbolically transform equation~\ref{eq:pade} into a normalized state-space model that avoids the need to compute any factorials.
We first do so for the special case of $p = q - 1$, since this provides the best approximation to the step-response~\citep{vajta2000some}.
We begin by expanding equation~\ref{eq:pade}:
\begin{align*}
[q-1/q]e^{-\theta s} &= \frac{\sum_{i=0}^{q-1} \begin{pmatrix}{q-1} \\ i\end{pmatrix} (2q - 1 - i)! (-1)^i \theta^i s^i}{\sum_{i=0}^q \begin{pmatrix}q \\ i\end{pmatrix} (2q - 1 - i)! \theta^i s^i} \\
&= \frac{\frac{1}{\theta^{q} (q-1)!} \sum_{i=0}^{q-1} \frac{(q-1)!}{(q-1-i)!i!} (2q - 1- i)! \theta^i s^i (-1)^i}{s^q + \frac{1}{\theta^q (q-1)!}  \sum_{i=0}^{q-1} \frac{q!}{(q-i)!i!} (2q - 1 - i)! \theta^i s^i} \\
&= \frac{\sum_{i=0}^{q-1} c_i s^i}{s^q + \sum_{i=0}^{q-1} d_i s^i} \text{,}
\end{align*}
where $d_i := \frac{q(2q - 1 - i)!}{(q-i)!i!} \theta^{i-q}$ and $c_i := (-1)^i \left( \frac{q-i}{q} \right) d_i$.

This transfer function is readily converted into a state-space model in controllable canonical form:
\begin{equation*}
    \begin{alignedat}{2}
        A &= \begin{pmatrix} -d_{q-1} & -d_{q-2} & \cdots & -d_0 \\ 1 & 0 & \cdots & 0 \\ 0 & \ddots & \ddots & \vdots \\ 0 & 0 & 1 & 0\end{pmatrix} \text{,} & & \quad\quad \begin{alignedat}{1}
            B &= \transpose{\begin{pmatrix} 1 & 0 & \cdots & 0\end{pmatrix}} \text{,} \\
            C &= \begin{pmatrix} c_{q-1} & c_{q-2} & \cdots & c_0\end{pmatrix} \text{,} \\
            D &= 0 \text{.}
        \end{alignedat}
    \end{alignedat}
\end{equation*}

To eliminate the factorials in $d_i$ and $c_i$, we scale the $i^{\text{th}}$ dimension of the state-vector by $d_{q-1-i}$, for all $i = 0 \ldots q - 1$.
This is achieved without changing the transfer function by scaling each $(B)_j$ by $d_{q-1-j}$, each $(C)_i$ by $1 / d_{q-1-i}$, and each $(A)_{ij}$ by $d_{q-1-i} / d_{q-1-j}$, which yields the equivalent state-space model:
\begin{equation*}
    \begin{alignedat}{2}
        A &= \begin{pmatrix} -v_0 & -v_0 & \cdots & -v_0 \\ v_1 & 0 & \cdots & 0 \\ 0 & \ddots & \ddots & \vdots \\ 0 & 0 & v_{q-1} & 0\end{pmatrix} \text{,} & & \quad\quad \begin{alignedat}{1}
            B &= \transpose{\begin{pmatrix} v_0 & 0 & \cdots & 0\end{pmatrix}} \text{,} \\
            C &= \begin{pmatrix} w_0 & w_1 & \cdots & w_{q-1} \end{pmatrix} \text{,} \\
            D &= 0 \text{,} \label{eq:ss-delay}
        \end{alignedat}
    \end{alignedat}
\end{equation*}
where $v_i := \frac{(q+i)(q-i)}{i+1} \theta^{-1}$ and $w_i := (-1)^{q - 1 - i} \left( \frac{i+1}{q} \right)$, for $i = 0 \ldots q-1$.
This follows from noting that $v_0 = d_{q-1}$ and $v_i := d_{q-1-i} / d_{q-i}$ for $i \ge 1$.

A similar derivation applies to the case where $p = q$, although it results in a passthrough ($D \ne 0$) which is suboptimal for step-responses.
For brevity, we omit this derivation, and instead simply state the result:
\begin{equation*}
    \begin{alignedat}{2}
        A &= \begin{pmatrix} -v_0 & -v_0 & \cdots & -v_0 \\ v_1 & 0 & \cdots & 0 \\ 0 & \ddots & \ddots & \vdots \\ 0 & 0 & v_{q-1} & 0\end{pmatrix} \text{,} & & \quad\quad \begin{alignedat}{1}
            B &= \transpose{\begin{pmatrix}-v_0 & 0 & \cdots & 0\end{pmatrix}} \text{,} \\
            C &= \begin{pmatrix} 2(-1)^q & 0 & 2(-1)^q & 0 & \cdots & \cdots \end{pmatrix} \text{,} \\
            D &= (-1)^q \text{,}
        \end{alignedat}
    \end{alignedat}
\end{equation*}
where $v_i = \frac{(q+i+1)(q-i)}{i+1} \theta^{-1}$, for $i = 0 \ldots q-1$.

In either case, $A$ and $B$ depend on the delay length solely by the scalar factor $\theta^{-1}$.
As a result, we may \emph{control} the length of the delay by adjusting the gain on the input and feedback signals.
The NEF can be used to build such controlled dynamical systems, without introducing multiplicative dendritic interactions or implausible on-the-fly connection weight scaling~\citep{eliasmith2000b}.
The identification of this control factor is connected to a more general property of the Laplace transform, $F \left( a^{-1} s \right) = \mathcal{L} \left\{ a f(at) \right\}$ for all $a > 0$, that we can exploit to modulate the width of any filter on-the-fly (in this case affecting the amount of delay; results not shown).

This model is now equivalent to equation~\ref{eq:pade}, but without any factorials, and in the form of equation~\ref{eq:lti}.\footnote{
In appendix~\ref{app:state-space}, we provide some additional manipulations of the state-space model.}
The choice of $q$ corresponds to the dimensionality of the latent state-vector $\V{x}(t)$ that is to be represented by Principle~1 and transformed by Principle~2.
Principle~3 may then be used to map equation~\ref{eq:ss-delay} onto a spiking dynamical network to accurately implement an optimal low-frequency approximation of the delay.

To demonstrate, we implement a $1$\,s delay of $1\,$Hz band-limited white noise using $\num{1000}$ recurrently connected spiking LIF neurons representing a $6$-dimensional vector space (see Figure~\ref{fig:delay-example}).
The connections between neurons are determined by applying Principle~3 (section~\ref{sec:principle3}) to the state-space model derived above (equation~\ref{eq:ss-delay}, $q=6$) via the Pad\'e approximants of the delay.
The normalized root-mean-squared error (NRMSE) of the output signal is $0.048$ (normalized so that $1.0$ would correspond to random chance).
This is achieved without appealing to the simulation time-step ($dt = 1$\,ms); in fact, as shown in section~\ref{sec:pure_delay}, the network accuracy improves as $dt$ approaches zero due to the continuous-time assumption mentioned in section~\ref{sec:principle3} (and resolved in section~\ref{sec:discrete-lowpass}).

\begin{figure}
  \centering
  \includegraphics[width=\textwidth]{{NECO-04-17-2838-Figure.3}.pdf}
  \caption{ \label{fig:delay-example}
    Delay of $1$\,s implemented by applying standard Principle~3 to equation~\ref{eq:ss-delay} using $q = 6$, $dt=1$\,ms, $\num{1000}$ spiking LIF neurons, and a lowpass synapse with $\tau=0.1$\,s.
    The input signal is white noise with a cutoff frequency of $1$\,Hz.
    The plotted spikes are filtered with the same $\tau=0.1$\,s, and encoded with respect to $\num{1000}$ encoders sampled uniformly from the surface of the hypersphere (sorted by time to peak activation).
  }
\end{figure}

Although the delay network has its dynamics optimized for a single delay $\theta > 0$, we can still accurately decode any delay $0 \le \theta' \le \theta$ from the same network.
This means that the network is representing a rolling window (i.e.,~history) of length $\theta$.
This window forms a temporal code of the input stimulus.

To compute these other delays, we would like to optimally approximate $e^{-\theta' s}$ with a transfer function $F_{\theta \rightarrow \theta'}(s) := \frac{\mathcal{C}(s; \, \theta, \theta')}{\mathcal{D}(s; \, \theta)}$ of order $[p / q]$, such that the denominator $\mathcal{D}(s; \, \theta)$ (which provides us with the recurrent transformation up to a change of basis) depends only on $\theta$, while the numerator $\mathcal{C}(s; \, \theta, \theta')$ (which provides us with the output transformation up to a change of basis) depends on some relationship between $\theta'$ and $\theta$.

From equation~\ref{eq:pade}, we may write the denominator as:
\begin{align*}
\mathcal{D}(s; \, \theta) = \sum_{i=0}^q d_i(\theta) s^i \text{,} \quad d_i(\theta) := \begin{pmatrix}q \\ i\end{pmatrix} \frac{(p + q - i)!}{(p + q)!} \theta^i \text{.}
\end{align*}
We then solve for the numerator, as follows:
\begin{align*}
&& [p/q] e^{-\theta' s} &= \sum_{i=0}^\infty \frac{(-\theta' s)^i}{i !} = \frac{\mathcal{C}(s; \, \theta, \theta')}{\mathcal{D}(s; \, \theta)} & \\
&& \iff \quad \mathcal{C}(s; \, \theta, \theta') &= \left( \sum_{i=0}^\infty \frac{(-\theta' s)^i}{i !} \right) \left( \sum_{j=0}^q d_j(\theta) s^j \right) + \mathcal{O}(s^{p + 1}) \text{.}
\end{align*}
By expanding this product and collecting like terms, the correct numerator up to order $p \le q$ is:
\begin{align*}
\mathcal{C}(s; \, \theta, \theta') = \sum_{i=0}^p c_i(\theta, \theta') s^i \text{,} \quad c_i(\theta, \theta') :=  \sum_{j=0}^i \frac{(- \theta')^{i - j}}{(i - j)!} d_j(\theta) \text{.}
\end{align*}
Therefore, the optimal readout for a delay of length $\theta'$, given the dynamics for a delay of length $\theta$, is determined by the above linear transformation of the coefficients $\left( d_j(\theta) \right)_{j=0}^p$.

We remark that $c_i(\theta, \theta) = \begin{pmatrix}p \\ i\end{pmatrix} \frac{(p + q - i)!}{(p + q)!} (-\theta)^i$, since $F_{\theta \rightarrow \theta}(s) = [p/q] e^{-\theta s}$, by uniqueness of the Pad\'e approximants, and by equation~\ref{eq:pade}.
As a corollary, we have proven that the following combinatorial identity holds for all $p, q \in \mathbb{N}$ and $i \in \left[ 0, \min\{p, q\} \right]$:
\begin{align*}
\begin{pmatrix}p \\ i\end{pmatrix} = \sum_{j=0}^i (-1)^j \begin{pmatrix}q \\ j\end{pmatrix} \begin{pmatrix}p + q - j \\ i - j\end{pmatrix} \text{.}
\end{align*}

For the case when $p = q - 1$, we may also apply the same state-space transformation from appendix~\ref{app:ss-delay} to obtain the normalized coefficients for the $C$ transformation (i.e.,~with $A$, $B$, and $D$ from equation~\ref{eq:ss-delay}):
\begin{align*}
w_{q-1-i} &= \left( \sum_{j=0}^i \frac{(-\theta')^{i-j}}{(i - j)!} \begin{pmatrix}q \\ j\end{pmatrix} \frac{(2q - 1 - j)!}{(2q - 1)!} \theta^j \right) \left( \frac{(q - i)! i! (2q - 1)!}{\theta^q (q - 1)! q(2q - 1 - i)!} \theta^{q - i} \right) \\
&= \sum_{j=0}^i \begin{pmatrix}q \\ j\end{pmatrix} \left( \frac{(2q - 1 - j)!}{(i - j)! (2q - 1 - i)!} \right) \left( \frac{(q - i)! i!}{q!} \right) \left( \theta^{j - i} \right) (-\theta')^{i - j} \\
&= \begin{pmatrix}q \\ i\end{pmatrix}^{-1} \sum_{j=0}^i \begin{pmatrix}q \\ j\end{pmatrix} \begin{pmatrix}2q - 1 - j \\ i - j\end{pmatrix} \left( \frac{-\theta'}{\theta} \right)^{i - j} \text{,} \quad i = 0 \ldots q - 1 \text{.}
\end{align*}

The $q$-dimensional state-vector of the delay network represents a rolling window of length $\theta$.
That is, a single delay network with some fixed $\theta > 0$ may be used to accurately decode any delay of length $\theta'$ ($0 \le \theta' \le \theta$).
Different decodings require different linear output transformations ($C$) for each $\theta'$, with the following coefficients:
\begin{align} \label{eq:delay-readouts}
w_{q-1-i} = \begin{pmatrix}q \\ i\end{pmatrix}^{-1} \sum_{j=0}^i \begin{pmatrix}q \\ j\end{pmatrix} \begin{pmatrix}2q - 1 - j \\ i - j\end{pmatrix} \left( \frac{-\theta'}{\theta} \right)^{i - j} \text{,} \quad i = 0 \ldots q - 1 \text{.} 
\end{align}
The underlying dynamical state remains the same.

% Apart from providing a means of understanding the state-vector, the basis functions from equation~\ref{eq:delay-readouts} also provide a number of ways to readily exploit the delay network.
% a) differentiate any point up to q times
% b) "preferred window" of each neuron
% c) projecting the window function onto state-space
% d) characterizing the possible functions using Principles 1+2 and the inverse basis functions

In Figure~\ref{fig:delay-full}, we take different linear transformations of the same state-vector, by evaluating equation~\ref{eq:delay-readouts} at various delays between $0$ and $\theta$, to decode the rolling window of input from the state of the system.\footnote{
The optimization problem from equation~\ref{eq:decoder_solution} need only be solved once to decode $\V{x}(t)$ from the neural activity.
The same decoders may then be transformed by each $C$ without loss in optimality (by linearity).
}
This demonstrates that the delay network compresses the input's history (lasting $\theta$ seconds) into a low-dimensional state.

\begin{figure}
  \centering
  \includegraphics[width=\textwidth]{{NECO-04-17-2838-Figure.4}.pdf}
  \caption{ \label{fig:delay-full}
    Decoding a rolling window of length $\theta$.
    Each line corresponds to a different delay, ranging from $0$ to $\theta$, decoded from a single delay network ($q = 12$).
    (Left)~Error of each delay, as the input frequency is increased relative to $\theta$.
    Shorter delays are decoded more accurately than longer delays at higher frequencies.
    A triangle marks $\theta = \text{Frequency}^{-1}$.
    (Right)~Example simulation decoding a rolling window of white noise with a cutoff frequency of $\theta^{-1}$\,Hz.
    % See appendix~\ref{app:window} for details.
  }
\end{figure}

In Figure~\ref{fig:basis-functions}, we sweep equation~\ref{eq:delay-readouts} across $\frac{\theta'}{\theta}$ to visualize the temporal ``basis functions'' of the delay network.
This provides a way to understand the relationship between the chosen state-space representation (i.e.,~the $q$-dimensional $\V{x}(t)$) and the underlying window representation (i.e.,~the infinite-dimensional $u(t)$).
In particular, each basis function corresponds to the continuous window of history represented by a single dimension of the delay network.
The instantaneous value of each dimension acts as a coefficient on its basis function, to contribute to the representation of the window at that point in time.
Overall, the entire state-vector determines a linear combination of these $q$ basis functions to represent the window.
This is analogous to the static function representation explored previously within the context of Principles~1 and~2~\citep[][pp.~63--72]{eliasmith2003a}.

\begin{figure}
  \centering
  \includegraphics[width=\textwidth]{{NECO-04-17-2838-Figure.5}.pdf}
  \caption{ \label{fig:basis-functions}
    Temporal basis functions of the delay network ($q = 12$).
    Each line corresponds to the basis function of a single dimension~($i$) ranging from $0$~(darkest) to $q - 1$~(lightest).
    The $i^\text{th}$ basis function is a polynomial over $\frac{\theta'}{\theta}$ with degree $i$ (see equation~\ref{eq:delay-readouts}). % ($0 \le \theta' \le \theta$).
    The state-vector of the delay network takes a linear combination of these $q$ basis functions in order to represent a rolling window of length $\theta$.
  }
\end{figure}

The encoder of each neuron can also be understood directly in these terms as taking a linear combination of the basis functions (via equation~\ref{eq:encoding}).
Each neuron nonlinearly encodes a projection of the rolling window onto some ``preferred window'' determined by its own encoder.
Since the state-vector is encoded by heterogeneous neural nonlinearities, the population's spiking activity supports the decoding of nonlinear functions across the entire window (i.e.,~functions that we can compute using Principles~1 and~2).
Therefore, we may conceptualize the delay network as a \emph{temporal coding} of the input stimulus, which constructs a low-dimensional state---representing an entire window of history---to encode the temporal structure of the stimulus into a nonlinear high-dimensional space of neural activities.

To more thoroughly characterize the delay dynamics, we analyze the behavior of the delay network as the dimensionality is increased (see Figure~\ref{fig:pca}).
Specifically, we perform a standard principal component analysis~(PCA) on the state-vector for the impulse response, and vary the order from $q=3$ to $q=27$.
This allows us to visualize a subset of the state-vector trajectories, via projection onto their first three principal components (see Figure~\ref{fig:pca}-Top). %\footnote{
%These same trajectories are obtained by a PCA on the %filtered neural activities, by linearity of decoding.
%}
The length of this trajectory over time distinguishes different values of $q$ (see Figure~\ref{fig:pca}-Bottom).
This length-curve is approximately logarithmic when $q = 6$, convex when $q \le 12$, and sigmoidal when $q > 12$. % (also see Figure~\ref{fig:time-cells}-Bottom).
To generate this figure we use a delay of $\theta = 10\,$s, but in fact this analysis is scale-invariant with time.
This means that other delays will simply stretch or compress the impulse response linearly in time (not shown).

\begin{figure}
  \centering
  \includegraphics[width=\textwidth]{{NECO-04-17-2838-Figure.6}.pdf}
  \caption{ \label{fig:pca}
    Impulse response of the delay network with various orders ($q$) of Pad\'e approximants.
    (Top)~The state-vector $\V{x}(t)$ projected onto its first three principal components.
    (Bottom)~The length of the curve $\V{x}$ up to time $t$, computed using the integral $\int_0^t \|\dot{\V{x}}(t')\| \, dt'$ (normalized to $1$ at $t = \theta$).
    This corresponds to the distance travelled by the state-vector over time.
    The dashed line marks the last inflection point, indicating when $\V{x}(t)$ begins to slow down.
    % The slope of $d(t)$ is correlated with the number of neurons encoding $\V{x}(t)$ at time $t$, when using a random encoding (not shown).
  }
\end{figure}

We remark that the delay network is scale-invariant with the delay length over input frequency, that is, the accuracy for a chosen order is a function of $s \times \theta$ (see units in Figure~\ref{fig:delay-full} for instance). % and the synaptic time-constants are scaled by $\theta$.
More specifically, for a fixed approximation error, the delay length scales as $\bigoh{ \frac{q}{f} }$, where $f$ is the input frequency.
Then, the accuracy of the mapped delay is a function of the relative magnitude of the delay length to $\frac{q}{f}$, whose exact shape depends on the considered synapse model.
To determine these functions for a wide class of synapses, we proceed by extending the NEF.


\section{Performance Comparison}

\section{Time Cell Comparison}
\label{sec:time-cells}

We now describe a connection between the delay network from section~\ref{sec:delay} and recent neural evidence regarding time cells.
Time cells were initially discovered in the hippocampus and proposed as temporal analogs of the more familiar place cells~\citep{eichenbaum2014}.
Similar patterns of neural activity have since been found throughout striatum~\citep{mello2015scalable} and cortex~\citep{luczak2015packet}, and have been extensively studied in the rodent mPFC~\citep{kim2013neural, tiganj2016sequential}.

Interestingly, we find that our delay network produces qualitatively similar neural responses to those observed in time cells.
This is shown in Figure~\ref{fig:time-cells}, by comparing neural recordings from mPFC~\citep[][Figure~4~C,D]{tiganj2016sequential} to the spiking activity from a network implementing a delay of the same length used in the original experiments.
Specifically, in this network, a random population of $300$ spiking LIF neurons maps a $4.784$\,s delay onto an alpha synapse ($\tau = 0.1$\,s) using our extension.
The order of the approximation is $q = 6$ (see equation~\ref{eq:ss-delay}), and the input signal is a rectangular pulse beginning at $t = -1$\,s and ending at $t = 0$\,s (height $= 1.5$).
The simulation is started at $t = -1$\,s and stopped at $t = 5$\,s.

\begin{figure}
  \centering
  \includegraphics[width=\textwidth]{{NECO-04-17-2838-Figure.10-Top}.pdf}
  \includegraphics[width=0.96\textwidth, trim=0 0 -0.7in -0.4in]{{NECO-04-17-2838-Figure.10-Bottom}.pdf}
  \caption{ \label{fig:time-cells}
    Comparison of time cells to a NEF delay network.
    (Top)~Spiking activity from the rodent mPFC~\citep[reproduced from][Figure~4~C,D]{tiganj2016sequential}.
    Neural recordings were taken during a maze task involving a delay period of $4.784$\,s.
    (Bottom)~Delay network implemented using the NEF (see text for details).
    %A random population of $300$ spiking LIF neurons map a $4.784$\,s delay onto an alpha synapse ($\tau = 0.1$\,s) using equation~\ref{eq:general-linear-approx}.
    %The order of the approximation is $q = 6$ (see equation~\ref{eq:ss-delay}), and the input signal is a rectangular pulse beginning at $t = -1$\,s and ending at $t = 0$\,s.
    $73$ time cells are selected by uniformly sampling encoders from the surface of the hypersphere.
    (A)~Cosine similarity between the activity vectors for every pair of time-points.
    The diagonal is normalized to the warmest colour.
    The similarity spreads out over time.
    (B)~Neural activity sorted by the time to peak activation.
    Each row is normalized between $0$ (cold) and $1$ (warm).
    We overlay the curve from Figure~\ref{fig:pca}-Bottom ($q = 6$) to model the peak-response times.
  }
\end{figure}

We also note a qualitative fit between the length-curve for $q=6$ in Figure~\ref{fig:pca} and the peak response-times in Figure~\ref{fig:time-cells}.
Specifically, Figure~\ref{fig:pca}-Bottom models the non-uniform distribution of the peak response-time of the cells as the length of the trajectory of $\V{x}(t)$ through time.
Implicit to this model are the simplifying assumptions that encoders are uniformly distributed, and that the L2-norm of the state-vector remains constant throughout the delay period.
Nevertheless, this model produces a qualitatively similar curve when $q = 6$ to both peak response-times from Figure~\ref{fig:time-cells}-Right (see overlay).

More quantitatively, we performed the same analysis on our simulated neural activity as \citet{tiganj2016sequential} performed on the biological data to capture the relationship between the peak and width of each time cell.
Specifically, we fit the spiking activity of each neuron with a Gaussian to model the peak time~($\mu_t$) and the standard deviation~($\sigma_t$) of each cell's ``time field''.\footnote{
We set $a_1 = P = S = 0$ in equation~1 from \citet{tiganj2016sequential}, since we have no external variables to control.
}
This fit was repeated for each of the $250$ simulated spiking LIF neurons that remained after selecting only those that had at least $90\%$ of their spikes occur within the delay interval.
The correlation between $\mu_t$ and $\sigma_t$ had a Pearson's coefficient of $0.68$ ($\rho < \num{e-34}$), compared to $0.52$ ($\rho < \num{e-5}$) for the biological time cells.
An ordinary linear regression model linking $\mu_t$ (independent variable) with $\sigma_t$ (dependent variable) resulted in an intercept of $0.27 \pm 0.06$ (standard error) and a slope of $0.40 \pm 0.03$ for our simulated data, compared to $0.27 \pm 0.07$ and $0.18 \pm 0.04$ respectively for the time cell data.
We note that we used the same bin size of $1$\,ms, modeled the same delay length, and did not perform any parameter fitting beyond the informal choices of $90\%$ cutoff, dimensionality ($q=6$), area of the input signal ($1.5$), and synaptic time-constant ($\tau = 0.1$\,s).

Neural mechanisms previously proposed to account for time cell responses have either been speculative~\citep{tiganj2016sequential},
or rely on gradually changing firing rates from a bank of arbitrarily long, ideally spaced, lowpass filters~\citep{shankar2012scale, howard2014unified, tiganj2015simple, tiganj2017neural}.
It is unclear if such methods can be implemented accurately and scalably using heterogeneous spiking neurons.
We suspect that robust implementation is unlikely given the high precision typically relied upon in these abstract models.
% For instance, the model from must be able to distinguish values exponentially close to $0$ in the neural representation.

In contrast, our proposed spiking model has its network-level dynamics derived from first principles to optimally retain information throughout the delay interval, without relying on a particular synapse model or bank of filters.
All of the neurons recurrently work together in a low-dimensional vector space to make efficient use of neural resources.
By using the methods of the NEF, this solution is inherently robust to spiking noise and other sources of uncertainty.
Furthermore, our explanation accounts for the nonlinear distribution of peak firing times as well as its linear correlation with the spread of time fields.

The observation of time cells across many cortical and subcortical areas suggests that the same neural mechanisms may be used in many circuits throughout the brain.
As a result, the neural activity implicated in a variety of delay tasks may be the result of many networks optimizing a similar problem to that of delaying low-frequency signals recurrently along a low-dimensional manifold.
Such networks would thus be participating in the temporal coding of a stimulus, by representing its history across a delay interval.
%\footnote{
%In appendix~\ref{app:ss-delay}, we briefly mention that this delay length may also be modulated on-the-fly.
% Some food for thought: since multiplication is so inaccurate, this might suggest that biological systems have a way to induce effective changes in the time-constants with some sort of adaptive normalization or gating or etc.
%In apendix~\ref{app:window}, we derive the output transformations required to decode a rolling window.
%}
% Possible predictions regarding time-constants / dimensionality?
% our network generalizes to predict the responses of time-cells to stimuli that are not simply discrete impulse events.


\section{Applications}
\label{sec:delay-applications}

\subsection{Delay Networks with Higher-order Synapses}
\label{sec:pure_delay}

We begin by making the practical point that it is crucial to account for the effect of the simulation time-step in digital simulations, if the time-step is not sufficiently small relative to the time scale of the desired network-level dynamics.
To demonstrate this, we simulate a $27$-dimensional delay network using $\num{1000}$ spiking LIF neurons, implementing a $0.1$\,s delay of $50$\,Hz band-limited white noise.
We vary the simulation time-step ($dt$) from $0.1$\,ms to $2$\,ms.
The accuracy of our extension does not depend on $dt$ (see Figure~\ref{fig:principle3fail}-Left).
When $dt=1$\,ms (the default in Nengo), the standard Principle~3 mapping (equation~\ref{eq:p3-novel}) obtains a NRMSE of $1.425$ ($43\%$ worse than random chance), versus $0.387$ for the discrete lowpass mapping which accounts for $dt$ (equation~\ref{eq:discrete-p3})---a $73\%$ reduction in error.
As $dt$ approaches $0$ the two methods become equivalent.

More to the point, we can analyze the delay network's frequency response
%\footnote{
%The frequency response is the transfer function evaluated at various input frequencies.
%}
when using a delayed continuous lowpass synapse (equation~\ref{eq:delayed-lowpass}) instead of the canonical lowpass (equation~\ref{eq:lowpass}) as the dynamical primitive.
This provides a direct measure of the possible improvement gains when using the extension.
Figure~\ref{fig:principle3fail}-Right compares the use of Principle~3 (which accounts for $\tau$ but ignores $\lambda)$, to our extension (which fully accounts for both; see section~\ref{sec:delayed-lowpass}) when $\lambda = \tau$.
The figure reveals that increasing the dimensionality improves the accuracy of our extension, while magnifying the error from Principle~3.
In the worst case, the Principle~3 mapping has an absolute error of nearly $\num{e15}$.
In practice, saturation from the neuron model bounds this error by the maximum firing rates.
Regardless, it is clearly crucial to account for axonal transmission delays to accurately characterize the network-level dynamics.

\begin{figure}
  \centering
  \includegraphics[width=1.0\textwidth]{{NECO-04-17-2838-Figure.8}.pdf}
  \caption{\label{fig:principle3fail}
    Comparing standard Principle~3 to our NEF extensions.
    (Left)~Error from mapping a $27$-dimensional $0.1$\,s delay onto $\num{1000}$ spiking LIF neurons, while varying the simulation time-step ($dt$).
    The input to the network is white noise with a cutoff frequency of $50$\,Hz.
    Unlike our extension, the standard form of Principle~3 does not account for $dt$.
    A dashed vertical line indicates the default time-step in Nengo.
    Error bars indicate a $95\%$ confidence interval bootstrapped across $25$ trials.
    (Right)~Mapping the delay system onto a delayed continuous lowpass synapse (with parameters $\frac{\tau}{\theta} = 0.1$ and $\frac{\lambda}{\tau} = 1$).
    The order of the delay system ($q$) is varied from $6$ (lightest) to $27$ (darkest).
    Each line evaluates the error in the frequency response, $\left| e^{-\theta s} - F^H(H(s)^{-1}) \right|$, where $F^H$ is determined by mapping the delay of order $q$ onto equation~\ref{eq:delayed-lowpass} using one of the two following methods.
    The method of our extension---which accounts for the axonal transmission delay---has monotonically increasing error that stabilizes at $1$ (i.e.,~the high frequencies are filtered).
    The standard Principle~3---which accounts for $\tau$ but ignores $\lambda$---alternates between phases of instability and stability as the frequency is increased.
  }
\end{figure}

To more broadly validate our NEF extensions from section~\ref{sec:extensions}, we map the delay system onto:
(1)~a continuous lowpass synapse (see section~\ref{sec:lowpass});
(2)~a delayed continuous lowpass synapse (see section~\ref{sec:delayed-lowpass}); and
(3)~a continuous double-exponential synapse (see section~\ref{sec:general}).
We apply each extension to construct delay networks of $\num{2000}$ spiking LIF neurons.
To compare the accuracy of each mapping, we make the time-step sufficiently small ($dt = 10\,\mu$s) to emulate a continuous-time setting.
We use the Pad\'e approximants of order $\left[5 / 6\right]$ for both equations~\ref{eq:pade} and~\ref{eq:lambert-delay}.
For the delayed lowpass, we again fix $\frac{\tau}{\theta} = 0.1$ and $\frac{\lambda}{\tau} = 1$.
For the double-exponential, we fix $\tau_1 = \tau$ and $\frac{\tau_1}{\tau_2} = 5$.
Expressing these parameters as dimensionless constants keeps our results scale-invariant with $\theta$.

\begin{figure}
  \centering
  \includegraphics[width=1.0\textwidth]{{NECO-04-17-2838-Figure.9}.pdf}
  \caption{\label{fig:lambert}
    The pure delay mapped onto spiking networks with various synapse models (with parameters $q = 6$, $\frac{\tau}{\theta} = 0.1$, $\frac{\lambda}{\tau} = 1$, $\tau_1 = \tau$, and $\frac{\tau_1}{\tau_2} = 5$).
    (Left)~Error of each mapping in the frequency domain.
    This subfigure is scale-invariant with $\theta$.
    (Right)~Example simulation when $\theta = 0.1\,$s and the input signal is white noise with a cutoff frequency of $15$\,Hz, corresponding to the triangle (over $1.5$) from the left subfigure.
    We use a time-step of $0.01$\,ms ($10\,\mu$s) and $\num{2000}$ spiking LIF neurons.
  }
\end{figure}

Figure~\ref{fig:lambert} reveals that axonal delays may be effectively ``amplified'' $10$-fold while reducing the NRMSE by $71\%$ compared to the lowpass (see Figure~\ref{fig:lambert}-Right; NRMSE for lowpass=$0.702$, delayed lowpass=$0.205$, and double-exponential=$0.541$).
The double-exponential synapse outperforms the lowpass, despite the additional poles introduced by the ZOH assumption in equation~\ref{eq:general-linear-approx} (see appendix~\ref{app:poles} for analysis).
This is because the double-exponential filters the spike-noise twice.
% Including the first-order derivative of the input signal further improves the double-exponential mapping by $X\%$.
Likewise, by exploiting an axonal delay, the same level of performance (e.g.,~$5\%$ error) may be achieved at approximately $1.5$ times higher frequencies, or equivalently for $1.5$ times longer network delays, when compared to the lowpass synapse (see~Figure~\ref{fig:lambert}-Left).
In summary, accounting for higher-order synaptic properties allows us to harness the axonal transmission delay to more accurately approximate network-level delays in spiking dynamical networks.

Together, these results demonstrate that our extensions can significantly improve the accuracy of high-level network dynamics.
Having demonstrated this for delays, in particular, suggests that the extension is useful for a wide variety of biologically relevant networks (see section~\ref{sec:time-cells}).

Detecting cyclic structure
Autocorelation (cosyne)
Storing and replaying episodic memories
Amplifying axonal spike delays
Amplifying one delay network into a larger delay network
