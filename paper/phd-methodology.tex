\chapter{Methodology}
\label{chapt:methodology}


\section{Software}

\subsection{Nengo}

Summarize key constructs.

\subsection{NengoLib}

Summarize main contributions.
 - ZOH LIF
 - Multi-spiking LIF
 - RK45
 - Dynamic extensions
 - Geometric extensions
 - Temporal learning (offline), RLS (online alternative to PES)
 - FORCE, Reservoir Computing

\subsection{Backends}

nengo\_dl, nengo\_ocl, nengo\_fpga, nengo\_brainstorm, nengo\_loihi


\section{Recurrent Architecture}

Resummarize Principle 3 with a focus on architecture / diagrams / pseudo-code.


\section{Optimization and Learning}

Some of this is going to be repetitive / overlapping with stuff from before, but I think they are the kind of things that are important to repeat...

We'll use this as an opportunity to point out the default scenario.

\subsection{Online versus Offline}

Online rules (PES, RLS), versus their offline analogs (stochastic gradient descent, least-squares)

\subsection{Explicit versus Implicit}

Using closed-form equations to generate the data (implicit), versus numerically simulating (explicit)

\subsection{Spikes versus Rates}

Learning from spikes versus learning from rates. And their equivalence in limiting conditions.


