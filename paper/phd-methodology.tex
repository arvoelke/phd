\chapter{Methodology}
\label{chapt:methodology}


\section{Software}

\subsection{Nengo Package}

Summarize key constructs.

Resummarize Principle 3 with a focus on architecture / diagrams / Nengo code.

\subsection{NengoLib Package}

Summarize main contributions.
 - ZOH LIF
 - Multi-spiking LIF
 - RK45
 - Dynamic extensions
 - Geometric extensions
 - Temporal learning (offline), RLS (online alternative to PES)
 - FORCE, Reservoir Computing

\subsection{Hardware Backends}

nengo\_dl, nengo\_ocl, nengo\_fpga, nengo\_brainstorm, nengo\_loihi


\section{Optimization and Learning}

Some of this is going to be repetitive / overlapping with stuff from before, but I think they are the kind of things that are important to repeat...

We'll use this as an opportunity to point out the default scenario.

\subsection{Online versus Offline}

Online rules (PES, RLS), versus their offline analogs (stochastic gradient descent, least-squares)

\subsection{Explicit versus Implicit}

Using closed-form equations to generate the data (implicit), versus numerically simulating (explicit)

\subsection{Spikes versus Rates}

Learning from spikes versus learning from rates. And their equivalence in limiting conditions.

\section{Dynamics as a Language}

Everything can be specified at a high-level as a high-dimensional nonlinear dynamical system.

\subsection{Dynamics of Learning}

Characterizing PES as a dynamical system

\subsection{Dynamics of Saturation}

Characterizing saturation as a dynamical system

\subsection{Dynamics of Error}

Writing down the dynamical system arising from the error term. e.g., oscillator error.

