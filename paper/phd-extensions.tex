\chapter{Extensions to the Neural Engineering Framework}
\label{chapt:nef-extensions}


\section{Challenges Posed by Neuromorphic Hardware}

Some of these are already solved to some extent by standard NEF. Others require more careful consideration / theoretical extensions.

\subsection{Discretization}

\subsection{Quantization}

\subsection{Connectivity}

\subsection{Memory}

Weight factorization on SpiNNaker, Braindrop, and Loihi

\subsection{Spike Traffic}

Amount of internal traffic.

\subsection{External Input-Output}

Amount of external traffic. Also encoding the inputs as spikes, and decoding the outputs.

\subsection{Thermal Variation}

\subsection{Transistor Mismatch}

\subsection{Digital to Analog Conversion}

Pulse-extender

\subsection{Higher-Order Dynamics}

Special case of transistor mismatch


\section{Arbitrary Linear Synapses}

Insert patent here.

\subsection{Linear Transfer Function Characterization}

\subsection{Nonlinear State-Space Characterization}

\subsection{Time-Constant Mismatch}

\subsection{Encoding Filters}

Tactile classification paper

\subsection{Decoding Filter Optimization}

Include method of simultaneously solving for decoders and linear filter in a single least-squares problem

\subsection{Dale's Principle}

Compare my DalesSolver to Parisian Transform

\section{Conductance-Based Synapses}

Summarize Andreas' research


\section{Biologically-Detailed Neurons}

\subsection{Adaptation}

Eric's system itendification
Subtractive adaptation and perfect cancellation
Divisive adaptation (adaptive threshold) and modelling it by the latter
Interpretation in terms of PSC code / objective function, and dimensionality

\subsection{Wilson Neurons}

Peter's work

\subsection{Baal Neurons}

Peter's work


\section{Energy Minimization}

``Differential encoding'' (adaptation)?

Spike-thinning (accumulator on BrainDrop)

Improving scaling by spreading out the spikes (inspired by Den\`eve)

Interneurons on Loihi

Other ideas: attentional routing (inhibition to the source). Dynamically adjusting firing rates from feedback error. Better automated selection of tuning curves (e.g., finding thresholds latent within the decomposition of functions)


\section{Dynamics of Learning}

Characterizing PES as a dynamical system


\section{Dynamics of Representational Error}

Writing down the dynamical system arising from the error term. e.g., oscillator error. Paying attention to saturation and making its effect precise.

